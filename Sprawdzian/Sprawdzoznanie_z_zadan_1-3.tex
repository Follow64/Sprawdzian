\documentclass[a4paper]{article}
% Kodowanie latain 2
%\usepackage[latin2]{inputenc}
\usepackage[T1]{fontenc}
% Można też użyć UTF-8
\usepackage[utf8]{inputenc}

% Język
\usepackage[polish]{babel}
% \usepackage[english]{babel}

% Rózne przydatne paczki:
% - znaczki matematyczne
\usepackage{amsmath, amsfonts}
% - wcięcie na początku pierwszego akapitu
\usepackage{indentfirst}
% - komenda \url 
\usepackage{hyperref}
% - dołączanie obrazków
\usepackage{graphics}
% - szersza strona
\usepackage[nofoot,hdivide={2cm,*,2cm},vdivide={2cm,*,2cm}]{geometry}
\frenchspacing
% - brak numerów stron
\pagestyle{empty}

% dane autora
\author{Jakub Fila}
\title{Raport z wykonania zadań 1-3}
\date{\today}
\begin{document}
\maketitle
\noindent
{\bf Zadanie 1.} \\
\noindent
Nastąpił problem z połączeniem z serwerem. Kolejne zadania wykonałem lokalnie na swoim komputerze. \\ \\
follow@DESKTOP-OT5CUTI:~/.ssh\$ ssh-copy-id mpyzik@pwi.ii.uni.wroc.pl \\
/usr/bin/ssh-copy-id: INFO: Source of key(s) to be installed: "/home/follow/.ssh/id\_rsa\_sprawdzian.pub" \\ 

/usr/bin/ssh-copy-id: INFO: attempting to log in with the new key(s), to filter out any that are already installed \\


/usr/bin/ssh-copy-id: ERROR: ssh: connect to host pwi.ii.uni.wroc.pl port 22: Resource temporarily unavailable \\
{\bf Zadanie 2.} \\
$\graphicspath{ {./zad1_cat,png} }$ \\
follow@DESKTOP-OT5CUTI:/mnt/c/Users/user\$ cat > Jakub\_Fila.txt  \\
\noindent
Jakub Fila \\
nr indeksu: 323060 \\
data: 24.11.2020 \\
follow@DESKTOP-OT5CUTI:/mnt/c/Users/user\$ seq 0 7 100 >> Jakub\_Fila.txt \\
follow@DESKTOP-OT5CUTI:/mnt/c/Users/user\$ mkdir testy \\
follow@DESKTOP-OT5CUTI:/mnt/c/Users/user\$ mv Jakub\_Fila.txt testy/ \\
follow@DESKTOP-OT5CUTI:/mnt/c/Users/user\$ cd testy/ \\
follow@DESKTOP-OT5CUTI:/mnt/c/Users/user/testy\$ ls \\
Jakub\_Fila.txt \\
$\\[1in]$
{\bf Zadanie 3.} \\
\noindent
Do sworzenia 5 plików użyłem komend: \\ \\
follow@DESKTOP-OT5CUTI:/mnt/c/Users/user/dane\$ hexdump /dev/urandom | head >> dane1.txt\\
follow@DESKTOP-OT5CUTI:/mnt/c/Users/user/dane\$ hexdump /dev/urandom | head >> dane2.txt\\
follow@DESKTOP-OT5CUTI:/mnt/c/Users/user/dane\$ hexdump /dev/urandom | head >> dane3.txt\\
follow@DESKTOP-OT5CUTI:/mnt/c/Users/user/dane\$ hexdump /dev/urandom | head >> dane4.txt\\
follow@DESKTOP-OT5CUTI:/mnt/c/Users/user/dane\$ hexdump /dev/urandom | head >> dane5.txt\\ \\
Do sklejenia plików: \\ \\
follow@DESKTOP-OT5CUTI:/mnt/c/Users/user/dane\$ cat dane1.txt dane2.txt dane3.txt dane4.txt dane5.txt > daneall.txt \\ \\
Nie doczytałem, że plik ma się nazywać contact, więc zmieniłem nazwę: \\ \\
follow@DESKTOP-OT5CUTI:/mnt/c/Users/user/dane\$ mv daneall.txt contact.txt \\ \\
Polecenie grep (akurat nie miałem żadnych linii które spełniałyby ten warunek): \\ \\ 
follow@DESKTOP-OT5CUTI:/mnt/c/Users/user/dane\$ grep '$\hat{}$0.*([a-z, 0-9][a-z, 0-9])$\backslash$\{2$\backslash$\}\$' contact.txt > output.txt \\
czytelniejniej wygląda screenshot ten komendy. \\
{\bf Nie starczyło mi czasu wkleić reszty komend, ale wszystko jest na screenshotach w folderze screeny\_terminala} \\
\end{document}